[1] F. Beritelli, and R. Grasso (2008), “A pattern recognition system for environmental sound classification based on MFCCs and neural networks," Proc. IEEE ICSPCS, pp.
1-4, Dec. 2008.
[2] L. Ma, D. Smith, and B. Milner (2003), "Enviromental
noise classification for context-aware applications,"
DEXA 2003, LNCS 2736, pp. 360-370.
[3] L. Couvreur, and M. Laniray (2004), "Automatic noise
recognition in urban environments based on artificial
neural networks and hidden markov models," Proc. 33rd
Inter-noise, Prague, Czech Republic.
[4]Sachin chachada, and C.-c. jay kuo(2014), "Environmental sound recognition: a survey," APSIPA Transactions on Signal and Information Processing , vol. 3, e14, page 1 of 15
[5]Juncheng Li, Wei Dai, Florian Metze, Shuhui Qu, Samarjit Das(2017),"A comparison of Deep Learning methods for environmental sound detection,"IEEE International Conference on Acoustics, Speech and Signal Processing (ICASSP)
[6]Guanghu Shen, Quang Nguyen, Jong SukChoi(2012),"An Environmental Sound Source Classification System Based on Mel-Frequency Cepstral Coefficients and Gaussian Mixture Models," IFAC Proceedings Volume 45, Issue 6,Pages 1802-1807
